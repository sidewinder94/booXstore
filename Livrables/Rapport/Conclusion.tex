\part{Conclusion et perspective}
\section{Recette}
\section{Analyse des écarts}
\section{Évolutions proposées}
\section{Bilans personnels et de groupe}

\begin{description}
	\item[Cyril LE ROY]
	\item[Antoine-Ali ZARROUK]
	\item[Benjamin ROBERT]
	Acclimatation obligatoire pour un environnement tel que le Java EE, où l'architecture d'une application légère peut prendre des allures d'application lourde seulement avec un langage.
Néanmoins, pas d'insatisfaction à la pensée de réaliser un site d'e-commerce pour la mise en application de ce langage.
De bonnes surprises quant aux différents outils utilisés, tels que BootStrap ou Git.
Une expérience trop courte par rapport aux cahier des charges qui peut rapidement être conséquent sur les petites fonctionnalités.
	\item[Julien NORMAND]
	Pour ce premier projet à l'EXIA.CESI, ça été une grande découverte. Dans un premier temps, il m'a fallu prendre connaissance du fonctionnement d'un projet, les documents que nous sommes amené à rendre ou à faire. Et la principale chose, comment était partagé le projet.
	Concernant le projet en lui même, il était interessant car on voyait plus en détails comment fonctionnait le JSF dans du Java EE. Vu mon niveau moyen en JAVA, je ne pensais pas parvenir à comprendre ce que je devais faire concernant ma partie. 
	\item[Jordan NOURRY]
	\item[Bilan de groupe]
\end{description}
