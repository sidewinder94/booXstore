\part{Conclusion et perspective}
\section{Recette}
\section{Analyse des écarts}
\section{Évolutions proposées}
\section{Bilans personnels et de groupe}

\begin{description}
	\item[Cyril LE ROY]
	Ce projet m’a appris deux choses. 
	La première, je n’aime pas Java EE. Quand on est habitué à développer des sites Web du même genre que BooXtore de manière simple et rapide, il est difficile de passer à la plateforme Java, à GlassFish et ses maudites erreurs de déploiement qui font perdre énormément de temps, aux chargements interminables et j’en passe. 
	Cependant, le fait de travailler en architecture MVC et d’être supporté par le langage Java (qui reste un langage très puissant et utile pour un développeur grâce à ses librairies riches) est très intéressant car cela se rapproche de nombreux Frameworks PhP et facilite donc de nombreuses étapes de développement.
	Evidemment, il est frustrant de passer beaucoup plus de temps à l’apprentissage et donc au développement sur des fonctionnalités qui semblent simples. 
	Pour conclure, je dirais que je viens tout juste de me rendre compte que depuis deux ans je me plains du manque de temps sur les projets et de ma frustration éternelle sur le fait de ne jamais pouvoir présenter un résultat digne de ce nom.
	\item[Antoine-Ali ZARROUK]
	\item[Benjamin ROBERT]
	Acclimatation obligatoire pour un environnement tel que le Java EE, où l'architecture d'une application légère peut prendre des allures d'application lourde seulement avec un langage.
Néanmoins, pas d'insatisfaction à la pensée de réaliser un site d'e-commerce pour la mise en application de ce langage.
De bonnes surprises quant aux différents outils utilisés, tels que BootStrap ou Git.
Une expérience trop courte par rapport aux cahier des charges qui peut rapidement être conséquent sur les petites fonctionnalités.
	\item[Julien NORMAND]
	Pour ce premier projet à l'EXIA.CESI, ça été une grande découverte. Dans un premier temps, il m'a fallu prendre connaissance du fonctionnement d'un projet, les documents que nous sommes amené à rendre ou à faire. Et la principale chose, comment était partagé le projet.
	Concernant le projet en lui même, il était interessant car on voyait plus en détails comment fonctionnait le JSF dans du Java EE. Vu mon niveau moyen en JAVA, je ne pensais pas parvenir à comprendre ce que je devais faire concernant ma partie. 
	\item[Jordan NOURRY]
	\item[Bilan de groupe]
	Ce projet a été un vrai challenge. Outre le fait de la complexité du projet pour à peine une semaine de développement, la conduite du projet aura été semée d’embuches. Tout d’abord, nous avons perdu un peu de temps en voulant utiliser un Framework de conception d’interfaces Web pour application Java appelé Vaadin. Au départ, les développeurs semblaient enthousiastes à l’idée d’utiliser ce Framework proposé par l’un des développeurs. Mais au vu de ces différentes utilisations de ce Framework, il semblait évident qu’il n’était pas du tout adapté à la conception de site Web tel que BooXtore. J’ai donc dû prendre la décision de rester sur JSF par sécurité, même si cela nous obligeait à reprendre une partie du travail. La deuxième embuche est arrivée au milieu du développement. Nos entités avaient été générées grâce à la base de données conçue précédemment mais des erreurs bloquantes insolvables nous empêchaient toute avancée. La décision a donc été prise de laisser tomber la base de données et de recréer nous-mêmes nos entités depuis le début, et ce, deux jours avant la fin de la phase de développement. Cette décision était certes un pari, et un pari gagnant car nous avons vite pu faire de grands progrès. Cependant, le goût amer du temps perdu sur un projet court a été assez démoralisant.
	Pour conclure, je ferais part de ma déception envers mon propre travail de chef de projet. J’étais vraiment motivé à conduire ce groupe de projet avec une bonne méthode de développement mais ce n’est que rendu à la fin que je me rends compte d’avoir peut être mal utilisé les ressources du projet et le temps alloué.

\end{description}
