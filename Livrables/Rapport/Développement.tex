\part{Développement}
\section{Justification des choix techniques}

	\subsection{Bootstrap}

		\href{http://getbootstrap.com/}{Bootstrap} est un framework CSS. Il s'agit d'un ensemble de règles définissant l'apparence de notre page.

		Utiliser Bootstrap nous permet de nous concentrer sur la structure de nos pages, plutôt que de créer nous-même nos styles alors que notre temps est déjà limité.

	\subsection{Git}

		\href{http://git-scm.com/}{Git} est un logiciel de gestion de versions. Il nous permet de gérer différentes versions concurrentes de notre code.

		Git nous a permis de travailler ensemble sur ce projet.

\section{Procédures de tests}
\section{Conception UML}
\section{Points forts et points faibles de notre implémentation}

	\subsection{Points forts}
	\textbf{La navbar :} Elle permet une utilisation rapide de la recherche de livre au lieu de voyager au travers des catégories afin d'avoir le livre qui nous convient le plus. La connexion/inscription de l'utilisateur est aussi présente sur celle-ci. Et pour conclure sur cette navbar, le panier affiche le pris et le nombre d'article de la commande en cours de l'utilisateur. Il lui reste ensuite qu'à cliquer dessus pour valider sa commande et passer au payement.

	\textbf{Le bootstrap :} Les pages crées avec ce style, permet d'avoir des pages intuitives pour les utilisateurs du store. L'esthétique/L'intuitivité du site est également important pour que l'utilisateur trouve ces repères rapidement et puisse ainsi revenir s'il trouve que le site lui plait. 
	Nos pages du store s'adapte également au fenêtre qu'on lui impose si elle est petite la navbar s'adapte à celle-ci et devient un bouton ou l'on clique dessus pour afficher le reste du menu.
 
	\subsection{Point faible}
	Le temps de chargement de la page de gestions des libraires est trop longue à s'ouvrir/charger vu le nombre de données qui doit être afficher sur celle-ci. 
	Cette page (management.xhmtl) peut-être optimisée pour devenir plusieurs pages au lieu de faire des onglets. En gros, un onglet devient une page. 